\documentclass[]{article}
\usepackage{lmodern}
\usepackage{amssymb,amsmath}
\usepackage{ifxetex,ifluatex}
\usepackage{fixltx2e} % provides \textsubscript
\ifnum 0\ifxetex 1\fi\ifluatex 1\fi=0 % if pdftex
  \usepackage[T1]{fontenc}
  \usepackage[utf8]{inputenc}
\else % if luatex or xelatex
  \ifxetex
    \usepackage{mathspec}
  \else
    \usepackage{fontspec}
  \fi
  \defaultfontfeatures{Ligatures=TeX,Scale=MatchLowercase}
\fi
% use upquote if available, for straight quotes in verbatim environments
\IfFileExists{upquote.sty}{\usepackage{upquote}}{}
% use microtype if available
\IfFileExists{microtype.sty}{%
\usepackage{microtype}
\UseMicrotypeSet[protrusion]{basicmath} % disable protrusion for tt fonts
}{}
\usepackage[margin=1in]{geometry}
\usepackage{hyperref}
\hypersetup{unicode=true,
            pdftitle={Dzień 4 - Modele nieliniowe, mieszane, GAM, GLM + miary niepodobieństwa - zadania},
            pdfauthor={Marcin K. Dyderski, Patryk Czortek},
            pdfborder={0 0 0},
            breaklinks=true}
\urlstyle{same}  % don't use monospace font for urls
\usepackage{color}
\usepackage{fancyvrb}
\newcommand{\VerbBar}{|}
\newcommand{\VERB}{\Verb[commandchars=\\\{\}]}
\DefineVerbatimEnvironment{Highlighting}{Verbatim}{commandchars=\\\{\}}
% Add ',fontsize=\small' for more characters per line
\usepackage{framed}
\definecolor{shadecolor}{RGB}{248,248,248}
\newenvironment{Shaded}{\begin{snugshade}}{\end{snugshade}}
\newcommand{\KeywordTok}[1]{\textcolor[rgb]{0.13,0.29,0.53}{\textbf{#1}}}
\newcommand{\DataTypeTok}[1]{\textcolor[rgb]{0.13,0.29,0.53}{#1}}
\newcommand{\DecValTok}[1]{\textcolor[rgb]{0.00,0.00,0.81}{#1}}
\newcommand{\BaseNTok}[1]{\textcolor[rgb]{0.00,0.00,0.81}{#1}}
\newcommand{\FloatTok}[1]{\textcolor[rgb]{0.00,0.00,0.81}{#1}}
\newcommand{\ConstantTok}[1]{\textcolor[rgb]{0.00,0.00,0.00}{#1}}
\newcommand{\CharTok}[1]{\textcolor[rgb]{0.31,0.60,0.02}{#1}}
\newcommand{\SpecialCharTok}[1]{\textcolor[rgb]{0.00,0.00,0.00}{#1}}
\newcommand{\StringTok}[1]{\textcolor[rgb]{0.31,0.60,0.02}{#1}}
\newcommand{\VerbatimStringTok}[1]{\textcolor[rgb]{0.31,0.60,0.02}{#1}}
\newcommand{\SpecialStringTok}[1]{\textcolor[rgb]{0.31,0.60,0.02}{#1}}
\newcommand{\ImportTok}[1]{#1}
\newcommand{\CommentTok}[1]{\textcolor[rgb]{0.56,0.35,0.01}{\textit{#1}}}
\newcommand{\DocumentationTok}[1]{\textcolor[rgb]{0.56,0.35,0.01}{\textbf{\textit{#1}}}}
\newcommand{\AnnotationTok}[1]{\textcolor[rgb]{0.56,0.35,0.01}{\textbf{\textit{#1}}}}
\newcommand{\CommentVarTok}[1]{\textcolor[rgb]{0.56,0.35,0.01}{\textbf{\textit{#1}}}}
\newcommand{\OtherTok}[1]{\textcolor[rgb]{0.56,0.35,0.01}{#1}}
\newcommand{\FunctionTok}[1]{\textcolor[rgb]{0.00,0.00,0.00}{#1}}
\newcommand{\VariableTok}[1]{\textcolor[rgb]{0.00,0.00,0.00}{#1}}
\newcommand{\ControlFlowTok}[1]{\textcolor[rgb]{0.13,0.29,0.53}{\textbf{#1}}}
\newcommand{\OperatorTok}[1]{\textcolor[rgb]{0.81,0.36,0.00}{\textbf{#1}}}
\newcommand{\BuiltInTok}[1]{#1}
\newcommand{\ExtensionTok}[1]{#1}
\newcommand{\PreprocessorTok}[1]{\textcolor[rgb]{0.56,0.35,0.01}{\textit{#1}}}
\newcommand{\AttributeTok}[1]{\textcolor[rgb]{0.77,0.63,0.00}{#1}}
\newcommand{\RegionMarkerTok}[1]{#1}
\newcommand{\InformationTok}[1]{\textcolor[rgb]{0.56,0.35,0.01}{\textbf{\textit{#1}}}}
\newcommand{\WarningTok}[1]{\textcolor[rgb]{0.56,0.35,0.01}{\textbf{\textit{#1}}}}
\newcommand{\AlertTok}[1]{\textcolor[rgb]{0.94,0.16,0.16}{#1}}
\newcommand{\ErrorTok}[1]{\textcolor[rgb]{0.64,0.00,0.00}{\textbf{#1}}}
\newcommand{\NormalTok}[1]{#1}
\usepackage{graphicx,grffile}
\makeatletter
\def\maxwidth{\ifdim\Gin@nat@width>\linewidth\linewidth\else\Gin@nat@width\fi}
\def\maxheight{\ifdim\Gin@nat@height>\textheight\textheight\else\Gin@nat@height\fi}
\makeatother
% Scale images if necessary, so that they will not overflow the page
% margins by default, and it is still possible to overwrite the defaults
% using explicit options in \includegraphics[width, height, ...]{}
\setkeys{Gin}{width=\maxwidth,height=\maxheight,keepaspectratio}
\IfFileExists{parskip.sty}{%
\usepackage{parskip}
}{% else
\setlength{\parindent}{0pt}
\setlength{\parskip}{6pt plus 2pt minus 1pt}
}
\setlength{\emergencystretch}{3em}  % prevent overfull lines
\providecommand{\tightlist}{%
  \setlength{\itemsep}{0pt}\setlength{\parskip}{0pt}}
\setcounter{secnumdepth}{0}
% Redefines (sub)paragraphs to behave more like sections
\ifx\paragraph\undefined\else
\let\oldparagraph\paragraph
\renewcommand{\paragraph}[1]{\oldparagraph{#1}\mbox{}}
\fi
\ifx\subparagraph\undefined\else
\let\oldsubparagraph\subparagraph
\renewcommand{\subparagraph}[1]{\oldsubparagraph{#1}\mbox{}}
\fi

%%% Use protect on footnotes to avoid problems with footnotes in titles
\let\rmarkdownfootnote\footnote%
\def\footnote{\protect\rmarkdownfootnote}

%%% Change title format to be more compact
\usepackage{titling}

% Create subtitle command for use in maketitle
\providecommand{\subtitle}[1]{
  \posttitle{
    \begin{center}\large#1\end{center}
    }
}

\setlength{\droptitle}{-2em}

  \title{Dzień 4 - Modele nieliniowe, mieszane, GAM, GLM + miary niepodobieństwa
- zadania}
    \pretitle{\vspace{\droptitle}\centering\huge}
  \posttitle{\par}
    \author{Marcin K. Dyderski, Patryk Czortek}
    \preauthor{\centering\large\emph}
  \postauthor{\par}
      \predate{\centering\large\emph}
  \postdate{\par}
    \date{4 kwietnia 2019}


\begin{document}
\maketitle

\subsection{Zadania do wykonania}\label{zadania-do-wykonania}

\begin{enumerate}
\def\labelenumi{\arabic{enumi}.}
\tightlist
\item
  Wczytaj zbiór danych z cechami sosen link:
  {[}\url{https://github.com/mkdyderski/BSS/blob/BSS2019/datasety/sosny.csv}{]}.
  Możesz również ściągnąć go do R za pomocą funkcji \texttt{read.csv()}:
\end{enumerate}

\begin{Shaded}
\begin{Highlighting}[]
\NormalTok{sosny<-}\KeywordTok{read.csv}\NormalTok{(}\StringTok{'https://raw.githubusercontent.com/mkdyderski/BSS/BSS2019/datasety/sosna.csv'}\NormalTok{,}
                \DataTypeTok{sep=}\StringTok{';'}\NormalTok{)}
\end{Highlighting}
\end{Shaded}

Używając funkcji \texttt{nls()} stwórz model nieliniowy biomasy części
nadziemnej \texttt{AB} o postaci \texttt{y=a*x\^{}b} jako funkcji
\texttt{Hg} używając formuł przedstawionych na wykładzie. Wykonaj wykres
używając pakietu \texttt{ggplot} i dodając linię modelu za pomocą
\texttt{geom\_smooth(method=\textquotesingle{}nls\textquotesingle{}....)},
pamiętaj o \texttt{SE=FALSE}.

\begin{enumerate}
\def\labelenumi{\arabic{enumi}.}
\setcounter{enumi}{1}
\tightlist
\item
  Wczytaj zbiór danych dotyczący występowania gatunków wskaźnikowych
  starych lasów w Poznaniu. Możesz również ściągnąć go do R za pomocą
  funkcji \texttt{read.csv()}:
\end{enumerate}

\begin{Shaded}
\begin{Highlighting}[]
\NormalTok{afis<-}\KeywordTok{read.csv}\NormalTok{(}\StringTok{'https://raw.githubusercontent.com/mkdyderski/BSS/BSS2019/datasety/afis.csv'}\NormalTok{,}
               \DataTypeTok{sep=}\StringTok{';'}\NormalTok{)}
\end{Highlighting}
\end{Shaded}

W zbiorze danych mamy informacje o udziale procentowym terenów otwartych
(agricultural, semi-natural \& wetlans, kolumna \texttt{ASW}), lasów
(\texttt{Forests}), terenów przemysłowych (\texttt{Industrial}), wód
('\texttt{Water}), zabudowy gęstej (\texttt{Urban.dense}) i rzadkiej
(\texttt{Urban.sparse}), typ lasów w kwadracie (\texttt{OLDFR},stare,
nowe i brak lasów), liczbę gatunków wskaźnikowych starych lasów
(\texttt{AFIS}) oraz obecność (0/1) pięciu wybranych gatunków.

\begin{enumerate}
\def\labelenumi{\alph{enumi}.}
\item
  Używając zbioru danych \texttt{afis} wykonaj model dla liczby gatunków
  wskaźnikowych starych lasów (kolumna \texttt{AFIS}) w opraciu o trzy
  predyktory: \texttt{Water}, \texttt{Urban.dense} oraz \texttt{OLDFR}.
  Z uwagi na charakter danych skorzystaj z rozkładu Poisson używając
  funkcji \texttt{glm(....,\ family=poisson)}
\item
  Wykonaj analogiczny model używając zamiast \texttt{OLDFR} kolumny
  \texttt{Forests}. Sprawdź który z modeli jest lepszy używając funkcji
  \texttt{AIC()}
\item
  Przygotuj wykres na którym pokażesz zależność pomiędzy AFIS a
  \texttt{Forests} z linią regresji zakładającą rozkład Poissona w
  oparciu o
  \texttt{geom\_smooth(method=\textquotesingle{}glm\textquotesingle{},method.args=list(family=\textquotesingle{}poisson\textquotesingle{}))}.
\end{enumerate}

\begin{enumerate}
\def\labelenumi{\arabic{enumi}.}
\setcounter{enumi}{2}
\tightlist
\item
  Korzystając ze zbioru danych \texttt{afis} przygotuj model
  występowania wybranego gatunku (np. \texttt{Ficavern}) używając jako
  predyktorów wybranych cech. Pamiętaj że występowanie gatunków w tym
  zbiorze danych jest wyrażone zerojedynkowo - użyj
  \texttt{glm(....,\ family\ =\ binomial(link=\textquotesingle{}logit\textquotesingle{}))}
\item
  Wczytaj zbiór danych \texttt{hotspots} link:
  {[}\url{https://github.com/mkdyderski/BSS/blob/BSS2019/datasety/hotspots.csv}{]}.
  Możesz również ściągnąć go do R za pomocą funkcji \texttt{read.csv()}:
\end{enumerate}

\begin{Shaded}
\begin{Highlighting}[]
\NormalTok{hotspots<-}\KeywordTok{read.csv}\NormalTok{(}\StringTok{'https://raw.githubusercontent.com/mkdyderski/BSS/BSS2019/datasety/hotspots.csv'}\NormalTok{,}
                   \DataTypeTok{sep=}\StringTok{';'}\NormalTok{)}
\end{Highlighting}
\end{Shaded}

Stwórz model mieszany (funkcja \texttt{lmer} z pakietu
\texttt{lmerTest}) bogactwa gatunkowego ptaków (kolumna \texttt{birds})
z efektami losowymi (\texttt{continent}) oraz stałymi (wybierz
interesujące Cię kolumny;) i za pomocą funkcji \texttt{r.squaredGLMM()}
z pakietu \texttt{MuMIn} sprawdź R2c i R2m.

\begin{enumerate}
\def\labelenumi{\arabic{enumi}.}
\setcounter{enumi}{4}
\tightlist
\item
  Wczytaj zbiór danych \texttt{survi} link:
  {[}\url{https://github.com/mkdyderski/BSS/blob/BSS2019/datasety/survi.csv}{]}.
  Możesz również ściągnąć go do R za pomocą funkcji \texttt{read.csv()}:
\end{enumerate}

\begin{Shaded}
\begin{Highlighting}[]
\NormalTok{survi<-}\KeywordTok{read.csv}\NormalTok{(}\StringTok{'https://raw.githubusercontent.com/mkdyderski/BSS/BSS2019/datasety/survi.csv'}\NormalTok{,}
                \DataTypeTok{sep=}\StringTok{';'}\NormalTok{)}
\end{Highlighting}
\end{Shaded}

W zbiorze tym sprawdź wpływ pH na przeżywalność siewek (kolumna
\texttt{surv}). Stwórz GLMM (funkcja \texttt{glmer} z pakietu
\texttt{lmerTest}) z rozkładem dwumianowym używając
\texttt{family=binomial(link=\textquotesingle{}logit\textquotesingle{})}
- jako efekt losowy sprawdź rok oraz blok - pomiń efekty związane z
plotem.

\begin{enumerate}
\def\labelenumi{\arabic{enumi}.}
\setcounter{enumi}{5}
\tightlist
\item
  Badano skład gatunkowy gatunków roślin runa na powierzchniach z: (i)
  usuniętym martwym świerkiem (litera „c'' przy id powierzchni), (ii)
  nieusuniętym martwym świerkiem zabitym przez kornika drukarza (litera
  „d'' przy id powierzchni) oraz (iii) drzewostanem nietkniętym przez
  gradację kornika (litera „f'' przy id powierzchni). Dane zawarto w
  pliku \texttt{dend.csv}. link:
  {[}\url{https://github.com/mkdyderski/BSS/blob/BSS2019/datasety/dend.csv}{]}.
  Możesz również ściągnąć go do R za pomocą funkcji \texttt{read.csv()}:
\end{enumerate}

\begin{Shaded}
\begin{Highlighting}[]
\NormalTok{dend<-}\KeywordTok{read.csv}\NormalTok{(}\StringTok{'https://raw.githubusercontent.com/mkdyderski/BSS/BSS2019/datasety/dend.csv'}\NormalTok{,}
                 \DataTypeTok{sep=}\StringTok{';'}\NormalTok{)}
\end{Highlighting}
\end{Shaded}

\begin{enumerate}
\def\labelenumi{\alph{enumi})}
\tightlist
\item
  Zaimportować dane do R
\item
  Za pomocą funkcji \texttt{vegdist()} w bibliotece \texttt{vegan}
  stworzyć macierz niepodobieństwa Bray-Curtisa pomiędzy próbami. Przed
  stworzeniem macierzy należy przeprowadzić transpozycję kolumn z
  wierszami.
\item
  Wyniki macierzy zobrazować za pomocą dendrogramów używając metody
  najbliższego i najdalszego sąsiada oraz metodą centroidów. Czy są
  obecne różnice? Która metoda okazuje się najlepsza do analizy
  powyższych danych?
\item
  Dendrogram najlepiej obrazujący wyniki
\end{enumerate}

\begin{itemize}
\tightlist
\item
  korzystając z biblioteki \texttt{ape} (jednocześnie pamiętając o
  konflikcie \texttt{ape} z \texttt{vegan}) zobrazować za pomocą różnych
  metod graficznych (\texttt{triangle}, \texttt{unrooted}, \texttt{fan}
  i \texttt{radial})
\item
  zmienić kolor, typ linii oraz kolor tekstu
\item
  podzielić na trzy klasy
\end{itemize}

\begin{enumerate}
\def\labelenumi{\alph{enumi})}
\setcounter{enumi}{4}
\tightlist
\item
  Korzystając z funkcji \texttt{beta.pair()} (biblioteka
  \texttt{betapart}) stworzyć macierz niepodobieństwa Sorensena pomiędzy
  próbami. Przed stworzeniem macierzy należy dokonać transformacji
  danych do postaci binarnej (przy użyciu funkcji
  \texttt{vegan::decostand()}).
\item
  Dane z macierzy zobrazować w postaci dendrogramu wybierając najlepszą
  strategię
\item
  Dla każdego typu powierzchni obliczyć średni wskaźnik różnorodności
  Shannona-Wienera i wskazać siedlisko o największej różnorodności
  gatunkowej. Dane id siedliska:
\end{enumerate}

\begin{Shaded}
\begin{Highlighting}[]
\NormalTok{id<-}\KeywordTok{c}\NormalTok{(}\KeywordTok{rep}\NormalTok{(}\StringTok{"clearcut"}\NormalTok{, }\DecValTok{30}\NormalTok{), }\KeywordTok{rep}\NormalTok{(}\StringTok{"dead"}\NormalTok{, }\DecValTok{19}\NormalTok{), }\KeywordTok{rep}\NormalTok{(}\StringTok{"forest"}\NormalTok{, }\DecValTok{25}\NormalTok{), }\KeywordTok{rep}\NormalTok{(}\StringTok{"dead"}\NormalTok{, }\DecValTok{2}\NormalTok{),}
      \StringTok{"forest"}\NormalTok{, }\KeywordTok{rep}\NormalTok{(}\StringTok{"dead"}\NormalTok{, }\DecValTok{3}\NormalTok{), }\KeywordTok{rep}\NormalTok{(}\StringTok{"forest"}\NormalTok{, }\DecValTok{4}\NormalTok{), }\KeywordTok{rep}\NormalTok{(}\StringTok{"dead"}\NormalTok{, }\DecValTok{4}\NormalTok{), }\StringTok{"forest"}\NormalTok{)}\StringTok{`}
\end{Highlighting}
\end{Shaded}

należy skleić kolumnami z obiektem zawierającym wskaźniki
Shannona-Wienera policzone dla każdej powierzchni w następujący sposób:

\begin{Shaded}
\begin{Highlighting}[]
\NormalTok{szanon<-}\KeywordTok{cbind}\NormalTok{(}\KeywordTok{as.data.frame}\NormalTok{(Shannon.index), id)}
\end{Highlighting}
\end{Shaded}

Wtedy średni wskaźnik Shannona-Wienera dla każdego typu powierzchni
można policzyć używając następującego kodu:

\begin{Shaded}
\begin{Highlighting}[]
\KeywordTok{mean}\NormalTok{(szanon}\OperatorTok{$}\NormalTok{Shannon.index[szanon}\OperatorTok{$}\NormalTok{id}\OperatorTok{==}\StringTok{"typ_siedliska"}\NormalTok{])}
\end{Highlighting}
\end{Shaded}

albo korzystając z pakietu \texttt{dplyr}:

\begin{Shaded}
\begin{Highlighting}[]
\KeywordTok{library}\NormalTok{(dplyr)}
\NormalTok{szanon}\OperatorTok\KeywordTok{group_by}\NormalTok{(id)}\OperatorTok\KeywordTok{summarise}\NormalTok{(}\DataTypeTok{m=}\KeywordTok{mean}\NormalTok{(Shannon.index))}
\end{Highlighting}
\end{Shaded}

\begin{enumerate}
\def\labelenumi{\alph{enumi})}
\setcounter{enumi}{7}
\tightlist
\item
  Używając tej samej procedury jak w podpunkcie (g), dla każdego typu
  powierzchni obliczyć średni wskaźnik równocenności Pielou. O czym mówi
  wskaźnik i czy są różnice pomiędzy trzema siedliskami?
\end{enumerate}

\subsection{Propozycje do pracy z własnym zbiorem
danych}\label{propozycje-do-pracy-z-wasnym-zbiorem-danych}

\begin{enumerate}
\def\labelenumi{\arabic{enumi}.}
\setcounter{enumi}{9}
\tightlist
\item
  Przetestuj hipotezy o wpływie czynników na zmienną zależną używając
  odpowiednich modeli. Weź pod uwagę rozkłady i logikę badanych
  zmiennych - np. tempo wzrostu korzeni nie może być ujemne, a
  temperatura ciała poniżej pewnej wartości oznacza śmierć.
\item
  Sprawdź czy do modelu należy włączyć efekty losowe - czasem może to
  przewrócić wnioskowanie do góry nogami, ale lepiej zinterpretować to
  teraz niż po uwagach recenzenta;) Zastanów się co może być
  modyfikowane przez czynniki losowe - nachylenie krzywej (tempo
  odpowiedzi) czy też tylko jej położenie (intercept)?
\item
  Jeśli korzystasz z analizy wariancji zastanów się czy nie włączyć do
  niej efektów losowych - spróbuj wrzucić w \texttt{anova()} obiekt typu
  \texttt{lmer} zamiast \texttt{lm}
\item
  Pracując na danych różnorodnościowych oblicz dla swoich danych
  wskaźniki różnorodności i porównaj ze swoimi wcześniejszymi
  obliczeniami (jeśli masz). Czy są jakieś różnice? Z czego mogą
  wynikać?
\item
  Czy można pogrupować obserwacje wg cech? Używając dendrogramów można
  np. sprawdzić czy obserwacje przyporządkowane do pewnych grup
  (jednostek fitosocjologicznych, lat, grup poletek, gatunków) grupują
  się wg tego klucza czy inaczej. Może się okazać, że np. fitosocjologia
  nie odzwierciedla rzeczywistości;)
\end{enumerate}


\end{document}
